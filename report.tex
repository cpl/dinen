\documentclass[a4paper]{article}
\begin{document}
\title{\Large\textbf{Team Project Final Report}}
\author {Oo Jia Shyang \\ Group X3}
\date {4/5/2017}
\maketitle
\newpage

\section{Introduction}

The following report would be a report about my experience participating in my team project. My team is team X3 and for our project, we have developed a web application called “Dinen” that resembles that of an online menu. It allows you to take orders online from restaurants. Therefore, it can function as a takeaway and at the same time work as an online menu when customers dine in. It’s like a cheaper version of McDonald’s build-in ordering interface. However, small restaurants that don’t have huge budget can’t afford that so our interface is perfect in this situation. They can use our website, register their restaurant online and they can start using tablets to take order in their restaurants as well. Not only is our website user-friendly to customers, but to restaurant owners too. It was my first time making a website from scratch, and a complicated one at that. I learned a lot and gained a lot during this experience.

\section{Tools and Techniques}\newline
I shall explain what tools I used during my project. I will talk about how I learned it, what they are, why and how I implemented it.
\newline

In this project, I worked on the front-end with 3 other team members. The tools that I used for coding are sublime text and atom. The languages that I utilized are Hypertext Markup Language (HTML), Cascading Style Sheets (CSS) and JavaScript (JS). We had a meeting about what languages to use and we all agreed that we would use these 3 instead of SASS or NodeJS as we wanted to learn the basics before using the slightly more advanced ones. First of all, I got to learn these languages online by myself as the school courses did not prioritise on these topics. I used multiple websites to see different syntaxes and how to implement them differently.I learned quite a lot online but knowing how to implement them efficiently and correctly is a different story. I was lucky enough that they were some semi-veterans of these languages in my team and they helped me throughout the project. Of course, google played a big part and stackoverflow was a very informational website regarding programming languages and problems that assisted me in my ordeals as well.
\newline

HTML is for making web pages and web applications. It is the backbone of the page. Whilst, CSS is for beautifying and decorating the page. For example, changing the fonts, adding pictures, aligning boxes and many more effects. These 2 languages always come in a pair as a HTML page without CSS would be too plain and boring. For the third one, JavaScript, it is a dynamic, high-level run-time language. In my opinion, it is the harder language of the three and it took me quite some time to understand it and implement it properly. JS can perform many functions including the client-side or even the server-side. It can be used to fix CSS errors on certain browsers, create highly responsive interfaces and much more. One of its awesome features is that it can load contents on the page when user needs it without refreshing the page, this is commonly known as Ajax. JS also allows us to requests data using JWT (Json Web Tokens) and link our requests to php. This is one of the more vital function that JS provides.
\newline

Despite having such amazing functions to these languages, being relatively new to these languages, we found it hard to make it mobile-friendly ourselves as it was very challenging. Therefore, we had to rely on a front-end framework called Bootstrap. It was extremely useful and essential to our website as it provided many classes that are flexible, mobile-friendly and functional in multiple browsers. We learned how to implement Bootstrap in our code and it worked like magic. The CSS classes that were provided were all responsive, beautiful and were easy to work with. Although bugs appear here and there occasionally, we worked together to fix them and learned from our mistakes. Besides that, we also used Bootstrap classes and id names when implementing JS function which make things easier for us as the names are unified.
\newline

\section{Concepts and principles that I have learned}\newline
I elaborate on some concepts that I have learned and principles that I have acquired.
\newline

One of the concepts that I have come to know is APIs. This is because I deemed it to be quite significant and a must-learn. One of the APIs that we used was restful (representational state transfer) API. It is an interface that uses HTTP requests to GET, PUT, POST and DELETE data. It made the process more efficient and convenient for developers, which is desirable. Just by implementing it, it helps with our project tremendously and makes the process smoother.
\newline

Secondly, a principle that I have acquired throughout the project was how important security and ethical issues are when developing a public website. For the security part, we implemented Hyper Text Transfer Protocol Secure (HTTPS). Using that, we ensured that communications between the browser and the user are encrypted. Thus, outsiders can’t simply obtain private and confidential information about our clients. In addition, we used SHA (Secure Hash Algorithm) 256 to hash our client’s passwords. This means that at no time would the password be shown clearly in the database or any hard drive. Thus, making our database secure, safe and reliable. Throughout the project, I have also realise that ethical and legal issues are actually quite essential when developing websites. Ethical issues played a big part in our website as we made sure we did not take copyrighted stuff, for example pictures. We wrote terms and conditions for our website to ensure rules and regulations are set in place. We made sure we did not do anything out of the law as being ignorant is not an excuse and it might lead to a big issue if not taken care of. 
\newpage


\section{Skills that I have developed}\newline
I talk about how I progressed as a member, what I have gained from working in a group and how this helps in me contributing to the project.
\newline

During the project, I have acquired multiple soft and technical skills. One of them is patience. As the saying goes, “patience is a virtue”. Being patient isn’t as easy as it seems. As I am still a greenhorn in the programming world, I made a lot of common mistakes and had to debug them very often. This ticked me off from time to time and made me procrastinate frequently. I ended up delaying my tasks and had to lengthen my deadlines that were set during group meetings. Besides that, I have become more open-minded to new things. Initially, I didn’t want to use interfaces and codes that were unfamiliar to me as I didn’t know if they were good and efficient. However, as time goes by, I noticed that these approaches are actually better, plus being more ubiquitous among veteran programmers. I learnt to be less stubborn and accept new things. Now, I investigate thoroughly about new methods or concepts before making a premature judgement and accepting them.
\newline

Moving on, I have become more proficient in the front-end languages that we deployed for our project. I have learnt that when you make web pages, you should not hard code the width or length of the design but instead, you make it responsive and cater to multiple browsers and devices. This was quite hard to learn and I have not master it yet. One of the achievements that I am proud of is writing a JS code that allows the each HTML pages to import its custom CSS file if it exists when the page is changed. Although the code wasn’t very complex, but it took me an entire night to make it work without any bugs. I was quite happy with it and that enable the members to make individual changes to each pages if needed as we were using a one-page architecture for our website.
\newline

Overall, these skills that I have acquired helped me improve my personal learning as I have become more enthusiastic in enquiring new knowledge. I became more adventurous and try out new things more often. This not only broadens my horizon of the language, it also deepens my understanding of its usages. It made me more capable and I could help the team more effectively. As a team member, I manage to increase my team’s efficiency as I can take part and help with more tasks now.
\newpage

\section{How effective did my team function}\newline
I will talk about our team management and team condition
\newline

Working as a team wasn’t an easy task, plus it was my first time doing a big project with a team. As a fresh team, it was good that we had a few very talented individuals in our team.
The entire team was enthusiastic and had good confidence in each other as we believe we could produce something amazing and refused to settle on an easy project. That was why we decided to create dinen. However, being the talented and more experienced ones, there were using many technical terms and complicated processes right off the bat and that put me off. They would assumed that these isn’t that hard and could easily be done. I was new and it was hard for me to keep up in this environment. Working together wasn’t easy at all. Luckily, they were patient enough to put up with me and explain those complicated processes to me. I also had to put in more effort in researching and getting to know about the subjects. I slowly picked up my pace and manage to work alongside them eventually. 
\newline

Initially, we had a bad communication problem and couldn’t delegate the tasks well. Many members use different communication platform. Fortunately, we found a common ground and decided to settle on “Slack”. From that day onwards, we posted our problems, progress and asked questions through Slack. It made our jobs easier and members more reachable. Sometimes, the team members slacked off and got lazy as time goes on. Thus, delaying the progression of the project and because of that, we decided to implement hard deadlines to ensure that the tasks that we have accepted is completed on time and a valid reason be given otherwise. On the front-end team, we made sure each of us created at least one page so that all of us get to employ our HTML and CSS skills. During weekly meetings, we give comments and suggestions to each other to improve the pages. So, each members would get to utilize and learn the necessary skills and contribute equally to the project. 
\newpage

\section{Future plans}\newline
I mention about the improvement and changes I want to achieve and some minor advices for future first years. 
\newline

First of all, I wasn’t extremely committed and hardworking when doing the project. I believe if I had put in a little more effort and hard work, I could have refined my skills more and contributed more to the team and finishing the more complicated things myself. Besides that, I tend to slack off now and then when I hit a roadblock while programming. Sometimes, this procrastination lasts up to 1 hour which is not good at all. Therefore, I aspire to be more hardworking and concentrate more when I do my work. I should solve the problem to the best of my ability instead of putting it off for another time.
\newline

As for the group management, we will not repeat the same mistake and just use the same communication platform from the start. This will ensure a smooth start and a good communication among team members right off the bat. Besides that, the team members should be more frank about their abilities and admit to their weaknesses so that we can all work more closely and efficiently.  
\newline

For the first years, I would strongly advice on using a unified communication platform immediately during their first meeting as that problem bugged us for a few weeks. It was a huge waste of time and annoying as members wouldn’t reply to our messages as we thought they use this platform when they actually don’t use it often. Besides that, be frank with each other and don’t slack off just because they have a good head start. It would come back to bite you hard in the end. Put your heart into the project, don’t do it just because you’re required to and be more open-minded and far-sighted.
\newline
\newline
\newline

\section{Conclusion}
I have learned a lot throughout this project and experience many hardships. I feel that I have become a better programmer and a better team worker as well. This has been a very refreshing experience to me as I have not work as team and committed as much in a long time. I still have many more to learn and will strive to do better and not lose my way in the coming years.
\newpage

\end{document}
